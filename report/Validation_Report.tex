\documentclass{article}

\usepackage[utf8]{inputenc}
\usepackage{longtable}
\usepackage[margin=1in]{geometry}

\def\code#1{\texttt{#1}}
% \csvtable{TEMPLATE}{MODEL}{FILE}
\newcommand\csvtable[3]{\begin{table}\caption{Validation data for #1 #2 #3}\footnotesize\csvautotabular[respect underscore=true]{../US/CA/SLAC/#1/autotest/models/gridlabd-4/#2/#3.csv}\end{table}}

\title
{
	HiPAS GridLAB-D Template Validation Data Addendum
}

\author
{
	David P. Chassin, Principal Investigator
\\
	SLAC National Accelerator Laboratory
\\
	Menlo Park, California (USA) 
}

\date { \today }

\begin{document}

\maketitle

\center 
{
	Copyright \copyright\ 2022, Regents of the Leland Stanford Junior University

\vspace{1cm}

	This report was created with funding from the California Energy Commission under grant EPC-17-046.

\vspace{1cm}

	SLAC National Accelerator Laboratory is operated by Stanford University for the US Department of Energy
	under Contract DE-AC02-67SF00515.
}

\newpage

\tableofcontents

\newpage

\listoftables

\newpage

\section{Background}

The HiPAS GridLAB-D Template repository \code{autotest} system runs all the published templates are all the models Version 4 models published on GitHuB. This report documents the validation data used when comparing the results of a proposed code change to the repository. Errors arising from this validation process will be reported as a \code{DIFF} error and cause a failure of the validation process.

\section{Loadfactor Template}

The \code{loadfactor} template allows loads in a model to be rescaled by an arbitrary factor. The validation test for this template rescales all the models by 50\%.

\input{loadfactor_loads.tex}

\section{ICA Analysis Template}

The \code{ica\_analysis} template allows Integration Capacity Analsys (ICA) to be performed on reference models. 

\input{ica_analysis_solar_capacity.tex}
\input{ica_analysis_violation_details.tex}

\end{document}
